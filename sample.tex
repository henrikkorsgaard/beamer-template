\documentclass[aspectratio=169]{beamer}
\usepackage[square,numbers]{natbib}
\bibliographystyle{apalike}
\title[Stibo Systems Case Presentation]{UX Architecture for Data Collaboration}
\date[]{\today}
\author[]{Henrik Korsgaard}
\institute{DEPARTMENT OF COMPUTER SCIENCE} % Type in A
\usetheme{henrikkorsgaard}

\usepackage{multicol}

\usetikzlibrary{shapes.geometric, shapes.symbols}
\usepackage{tikzpeople}
\usetikzlibrary{arrows.meta}
\usepackage{fontawesome5}
\usepackage{changepage}

\makeatletter
\newbox\@backgroundblock
\newenvironment{backgroundblock}[2]{%
  \global\setbox\@backgroundblock=\vbox\bgroup%
    \unvbox\@backgroundblock%
    \vbox to0pt\bgroup\vskip#2\hbox to0pt\bgroup\hskip#1\relax%
}{\egroup\egroup\egroup}
\addtobeamertemplate{background}{\box\@backgroundblock}{}
\makeatother

\makeatletter
    \newsavebox\restorebox
\newenvironment{restoretext}%
    {\@parboxrestore% 
     \begin{adjustwidth}{-8mm}{-8mm}%
                \begin{lrbox}{\restorebox}%
                \begin{minipage}{\linewidth}%
    }{\end{minipage}\end{lrbox}
        \usebox\restorebox
        \end{adjustwidth}
     }
\makeatother

\begin{document}

\begin{frame}
\maketitle
\end{frame}

\begin{frame}{Outline}
    \begin{columns}
        \begin{column}{0.5\textwidth}
            \begin{itemize}
                \item UX research process
                \item Results and observations
                \item Information design
                \item Interaction design
                %%\item If time, responsive filtering %% if you cannot execute this within time, then keep
            \end{itemize}
        \end{column}
        \begin{column}{0.5\textwidth}
            As a newly hired UX architect, your initial task is to create an outline for the UX work in a project aimed at improving the UX of the \textbf{collaboration tooling}\footnotemark{} in an existing online Excel-like table system.\\
            \vspace{1em}
            {\color{HenrikFontLight}\dots assume that you have the necessary budget for it.}
        \end{column}
    \end{columns}
    \footnotetext{Ida Larsen-Ledet, and Henrik Korsgaard. Territorial functioning in \textbf{collaborative writing}. CSCW 2019\\Ida Larsen-Ledet, Henrik Korsgaard, and Susanne Bødker. \textbf{Collaborative writing} across multiple artifact ecologies.CHI 2020}
\end{frame}

\begin{frame}{UX Research - key questions}
    \vspace{1em}
    \begin{columns}
        \begin{column}{0.6\textwidth}
            \textbf{Why, how, and when do users collaborate?}\\
            \vspace{1em}
            \begin{itemize}
                \small
                \item What are the primary tasks and objectives?
                \item What are the different roles and responsibilities in collaboration? 
                \item How do remote work impact the user experience?
                \item What other tools do they use to support the tasks -- communication, analysis etc.?
            \end{itemize}
        \end{column}
        \begin{column}{0.4\textwidth}
            \begin{figure}[h]
                \centering
                \includegraphics[width=0.8\textwidth]{images/collaborators.png}
            \end{figure}
        \end{column}
    \end{columns}
%     \begin{restoretext}
%     \begin{backgroundblock}{7mm}{50mm}
%     \centering
%     \resizebox{1\textwidth}{!}{%
%     \begin{tikzpicture}

%         \node[diamond,fill=HenrikDark!4, minimum width = 4.5cm, minimum height = 4.5cm] (d) at (1.1,1) {};
%         \node[diamond,fill=HenrikDark!4, minimum width = 4.5cm, minimum height = 4.5cm] (d) at (5.5,1) {};

%         \node[signal to=east, signal from=west,shape=signal, draw, minimum width = 2.15cm, minimum height = 0.5cm, fill=white] at (0.1,1) {\tiny Discover};
%         \node[signal to=east, signal from=west, shape=signal, draw, minimum width = 2.3cm, minimum height = 0.5cm, fill=white] at (2.2,1) {\tiny Define};
%         \node[signal to=east, signal from=west, shape=signal, draw, minimum width = 2.3cm, minimum height = 0.5cm, fill=white] at (4.37,1) {\tiny Prototype};
%         \node[signal to=east, signal from=west, shape=signal, draw, minimum width = 2.3cm, minimum height = 0.5cm, fill=white] at (6.54,1) {\tiny Evaluate};
%         \node[signal to=east, signal from=west, shape=signal, draw, minimum width = 2.3cm, minimum height = 0.5cm, fill=white] at (8.7,1) {\tiny Integrate}; 
%     \end{tikzpicture}
%     }
% \end{backgroundblock}
% \end{restoretext}
\end{frame}


\begin{frame}{UX Research}
    \vspace{-1em}
    \begin{columns}[t]
        \begin{column}{0.5\textwidth}
            \small
            \textbf{Discover}
            \begin{enumerate}
                \item Observe collaborative session
                \item Contextual interviews
                \item Internal/external research
                \item Analytics and in-app surveys
                \item Workshops
            \end{enumerate}
        \end{column}
        \begin{column}{0.5\textwidth}
            \small
            \textbf{Define}
            \begin{itemize}
                \item Collaborative task objectives 
                \item Scenarios, personas and user journeys
                \item Information concepts and architecture
                \item UX quality criteria and KPIs
            \end{itemize}
        \end{column}
    \end{columns}
    \begin{restoretext}
    \begin{backgroundblock}{7mm}{50mm}
    \centering
    \resizebox{1\textwidth}{!}{%
    \begin{tikzpicture}

        \node[diamond,fill=HenrikDark!4, minimum width = 4.5cm, minimum height = 4.5cm] (d) at (1.1,1) {};
        \node[diamond,fill=HenrikDark!4, minimum width = 4.5cm, minimum height = 4.5cm] (d) at (5.5,1) {};

        \node[signal to=east, signal from=west,shape=signal, draw, minimum width = 2.15cm, minimum height = 0.5cm,white, fill=HenrikDark] at (0.1,1) {\tiny Discover};
        \node[signal to=east, signal from=west, shape=signal, draw, minimum width = 2.3cm, minimum height = 0.5cm,white,fill=HenrikDark] at (2.2,1) {\tiny Define};
        \node[signal to=east, signal from=west, shape=signal, draw, minimum width = 2.3cm, minimum height = 0.5cm] at (4.37,1) {\tiny Prototype};
        \node[signal to=east, signal from=west, shape=signal, draw, minimum width = 2.3cm, minimum height = 0.5cm, fill=white] at (6.54,1) {\tiny Evaluate};
        \node[signal to=east, signal from=west, shape=signal, draw, minimum width = 2.3cm, minimum height = 0.5cm, fill=white] at (8.7,1) {\tiny Integrate}; 
    \end{tikzpicture}
    }
\end{backgroundblock}
\end{restoretext}
\end{frame}

\begin{frame}{UX Research -- iterate where needed}
    \vspace{-1em}
    \begin{columns}[t]
        \begin{column}{0.5\textwidth}
            \small
            \textbf{Prototype}
            \begin{itemize}
                \item Collaborative user flow
                \item Information architecture and UI design
                \item Key UI components and technical features
            \end{itemize}
        \end{column}
        \begin{column}{0.5\textwidth}
            \small
            \textbf{Evaluate}
            \begin{itemize}
                \item Internal review and testing
                \item User feedback (informal/think aloud)
                \item Review UX quality criteria and KPIs
            \end{itemize}
        \end{column}
    \end{columns}
    \begin{restoretext}
    \begin{backgroundblock}{7mm}{50mm}
    \centering
    \resizebox{1\textwidth}{!}{%
    \begin{tikzpicture}

        \node[diamond,fill=HenrikDark!4, minimum width = 4.5cm, minimum height = 4.5cm] (d) at (1.1,1) {};
        \node[diamond,fill=HenrikDark!4, minimum width = 4.5cm, minimum height = 4.5cm] (d) at (5.5,1) {};

        \node[signal to=east, signal from=west,shape=signal, draw, minimum width = 2.15cm, minimum height = 0.5cm] at (0.1,1) {\tiny Discover};
        \node[signal to=east, signal from=west, shape=signal, draw, minimum width = 2.3cm, minimum height = 0.5cm] at (2.2,1) {\tiny Define};
        \node[signal to=east, signal from=west, shape=signal, draw, minimum width = 2.3cm, minimum height = 0.5cm, white, fill=HenrikDark] at (4.37,1) {\tiny Prototype};
        \node[signal to=east, signal from=west, shape=signal, draw, minimum width = 2.3cm, minimum height = 0.5cm, white, fill=HenrikDark] at (6.54,1) {\tiny Evaluate};
        \node[signal to=east, signal from=west, shape=signal, draw, minimum width = 2.3cm, minimum height = 0.5cm, fill=white] at (8.7,1) {\tiny Integrate}; 
    \end{tikzpicture}
    }
\end{backgroundblock}
\end{restoretext}
\end{frame}

% \begin{frame}{Results: Collaborative scenarios}
%     \begin{columns}
%         \begin{column}{0.7\textwidth}
%             \small
%             \textbf{Collaboration projects}
%             \begin{itemize}
%                 \footnotesize
%                 \item Multiple peers collaborate on a larger task over time
%                 \item Different roles, responsibilities and expertise
%                 \item Turn-taking with a high degree of coordination
%             \end{itemize}
%             \textbf{Incident response}
%             \begin{itemize}
%                 \footnotesize
%                 \item 2 peers collaborate to solve a pressing issue
%                 \item Real-time ``jump-in-and-consult'' collaboration
%             \end{itemize}
%             \textbf{Training}
%             \begin{itemize}
%                 \footnotesize
%                 \item Instructor help one or more trainees
%                 \item Focused on learning the application and/or data 
%             \end{itemize}
%         \end{column}
%         \begin{column}{0.3\textwidth}
%             \centering
%             \resizebox{0.8\textwidth}{!}{%
%             \begin{tikzpicture}
%                 \node[] at (0,2.1) {\small\faUser};
%                 \draw[{Straight Barb[scale=0.5]}-{Straight Barb[scale=0.5]}] (0.25,2) to[bend right=-45] (0.58,1.5);
%                 \node[] at (0.55,1.2) {\small\faUser};
%                 \draw[{Straight Barb[scale=0.5]}-{Straight Barb[scale=0.5]}] (-0.25,2) to[bend right=45] (-0.58,1.5);
%                 \node[] at (0,1.5) {\scriptsize\faDatabase};
%                 \draw[{Straight Barb[scale=0.5]}-{Straight Barb[scale=0.5]}] (-0.3,1) to[bend right=45] (0.3,1);
%                 \node[] at (-0.55,1.2) {\small\faUser};

%                 \node[] at (0.7,0) {\small\faUser};
%                 \draw[{Straight Barb[scale=0.5]}-] (0.15,0) to (0.5,0);
%                 \node[] at (0,0) {\tiny\faDatabase};
%                 \draw[{Straight Barb[scale=0.5]}-] (-0.15,0) to (-0.5,0);
%                 \node[] at (-0.7,0) {\small\faUser};

%                 \node[] at (-0.7,-1.5) {\small\faUser};
%                 \draw[-{Straight Barb[scale=0.5]}] (-0.5,-1.5) to (-0.15,-1.5);
%                 \node[] at (0,-1.5) {\tiny\faDatabase};
%                 \draw[-{Straight Barb[scale=0.5]}] (0.15,-1.4) to (0.5,-1.1);
%                 \draw[-{Straight Barb[scale=0.5]}] (0.15,-1.5) to (0.5,-1.5);
%                 \draw[-{Straight Barb[scale=0.5]}] (0.15,-1.6) to (0.5,-1.9);
%                 \node[] at (0.7,-1.1) {\scriptsize\faUser};
%                 \node[] at (0.7,-1.5) {\scriptsize\faUser};
%                 \node[] at (0.7,-1.9) {\scriptsize\faUser};
%             \end{tikzpicture}
%             }
%         \end{column}
%     \end{columns}
% \end{frame}

\begin{frame}{Results: Collaborative scenarios}
    \setlength{\leftmargini}{0.5cm}
    \begin{columns}[t]
        \begin{column}{0.33\textwidth}
            \textbf{\small 1. Collaborative projects}
            \begin{itemize}
                \scriptsize
                \item Peers collaborate on a larger project
                \item Different responsibilities and expertise
                \item Mixed focus collaboration with a high degree of coordination
                \item Multiple data views
            \end{itemize}
        \end{column}
        \begin{column}{0.33\textwidth}
            \textbf{\small 2. Real-time collaboration}
            \begin{itemize}
                \scriptsize
                \item Peers collaborate on smaller (urgent) tasks
                \item Real-time collaboration
                \item Shared task focus 
                \item Few data views
            \end{itemize}
        \end{column}
        \begin{column}{0.33\textwidth}
            \textbf{\small 3. Training}
            \begin{itemize}
                \scriptsize
                \item Expert user provide training and onboarding of novices
                \item Focused on learning the application and/or data
                \item Tailored data views and exercises
            \end{itemize}
        \end{column}
    \end{columns}
\end{frame}

\begin{frame}{Results: Key UX qualities}
    \begin{columns}
        \begin{column}{0.6\textwidth}
            \small
            \begin{itemize}
                \item Sharing with collaborators should be easy and include task assignment and notes
                \item Important to know who did what in a shared view (awareness, track changes, accountability etc.)
                \item Support sandbox experimentation and analysis before publishing
                \item Collaborative features should not overshadow existing task features
                %\item A majority mention Google Drive as an example of a great collaboration tool
            \end{itemize}
        \end{column}
        \begin{column}{0.4\textwidth}
            \begin{figure}[h]
                \centering
                \includegraphics[width=1\textwidth]{images/Users.png}
            \end{figure}
        \end{column}
    \end{columns}
\end{frame}

% \begin{frame}{Results: State-of-the-art collaborative UX}
%     \begin{columns}
%         \begin{column}{0.6\textwidth}
%             \begin{itemize}
%                 \small
%                 \item Work object is shared by replication (content and formatting)
%                 \item Communication is transient (chat)
%                 \item Tools are individual, but similar across users (`what-you-see-I-see')
%                 \item Environment is not shared (browser/extensions)
%                 \item[]
%                 \item Not perfect, e.g. `the jumping text problem' and `cursor wars'
%             \end{itemize}
%         \end{column}
%         \begin{column}{0.4\textwidth}
%             \begin{figure}[h]
%                 \centering
%                 \includegraphics[width=1\textwidth]{images/gdocs.png}
%                 %\caption{Google Documents Sharing Model}
%             \end{figure}
%         \end{column}
%     \end{columns}
% \end{frame}

\begin{frame}{Interaction Design features}
    \begin{itemize}
        \item The user can save changes as individual \textbf{views} (sheets) of data
        \item \textbf{The user can share their saved views with other users}
        \item[] $\rightarrow$ The user can add or remove columns from the \textbf{view}
        \item[] $\rightarrow$ Users can filter and order the table content
        \item \textbf{Multiple users must be able to work on the same views simultaneously}
        \item[] $\rightarrow$ The users of the system may be located on multiple locations
    \end{itemize} 
\end{frame}


\begin{frame}{Information Design: Data views as first class objects}
    \begin{columns}
        \begin{column}{0.4\textwidth}
            %% sige noget om hvorfor det er en challenge
            \begin{itemize}
                \footnotesize
                \item The main information artifact when collaborating -- it's what we share and work on
                \item A data view encapsulate a data source, users, and the revision history
                \item Views can be published in formats fitting the consumer needs
            \end{itemize}
        \end{column}
        \begin{column}{0.6\textwidth}
            \begin{figure}[h]
                \centering
                \includegraphics[width=1.1\textwidth]{images/all-views-with-marks.png}
            \end{figure}
        \end{column}
    \end{columns}
\end{frame}

\begin{frame}{Creating a new data view}
    \begin{figure}[h]
        \centering
        \includegraphics[width=1\textwidth]{images/create-new-view-flow.png}
    \end{figure}
\end{frame}

\begin{frame}{Collaborative tooling with tabular data}
    \begin{columns}
        \begin{column}{0.4\textwidth}
            %% sige noget om hvorfor det er en challenge
            \begin{itemize}
                \small
                %\item Tabular view is the primary synchronized/shared object
                %\item \dots filters cannot be shared until operationalized
                \item \textbf{Action items:} to support different roles and tasks
                \item \textbf{History} to support track changes and accountability
                \item \textbf{Collaborator} pane for awareness and navigation
            \end{itemize}
        \end{column}
        \begin{column}{0.6\textwidth}
            \begin{figure}[h]
                \centering
                \includegraphics[width=1\textwidth]{images/filter-view-with-marks.png}
            \end{figure}
        \end{column}
    \end{columns}
\end{frame}

\begin{frame}{Assign action item to collaborators}
    \vspace{2em}
    \begin{restoretext}
    \begin{figure}[h]
        \centering
        \includegraphics[width=1\textwidth]{images/assign-action-item.png}
    \end{figure}
\end{restoretext}
\end{frame}

\begin{frame}{Revision history as key\\ collaboration support}
    \begin{columns}
        \begin{column}{0.5\textwidth}
            \begin{itemize}
                \small
                \item Data operations as the replicated objects (CRDT)
                \item Support task resumption, accountability and finding stuff
                \item Support experimentation -- you can always roll back changes
                \item A set of operations can be applied to other data views (macros)
            \end{itemize}
        \end{column}
        \begin{column}{0.5\textwidth}
            \begin{figure}[h]
                \vspace{-4em}
                \centering
                \includegraphics[width=0.9\textwidth]{images/shared-history.png}
            \end{figure}
        \end{column}
    \end{columns}
\end{frame}


%% Skip if needed
% \begin{frame}{Key IA/IxD challenges}
%     \setlength\itemsep{10em}
%     \begin{itemize}
%         \item `Views' might be a difficult concept to grasph 
%         \item Revision history is powerful tool, but difficult to get right 
%         \item Remote collaboration \textbf{require} additional communication channels
%         \item Developing and prototyping collaborative features pose different requirements than single user applications:
%         \begin{itemize}
%             \footnotesize
%             \vspace{0.3em}
%             \item Require more contextual user research and co-design activities
%             \vspace{0.1em}
%             \item Hard to study from application analytics
%             \vspace{0.1em}
%             \item Require some technical infrastructure to prototype collaboration
%         \end{itemize}
%     \end{itemize}
% \end{frame}

\begin{frame}[DarkSlide]{}
    \vspace{3em}
    \centering
    \Large THANK YOU\\
\end{frame}

\end{document}